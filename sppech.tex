% filepath: /home/zhao/Downloads/my-threejs-project/SECM_DEMO_JSX/speech.tex
\documentclass{beamer}
\usetheme{Madrid}
\usecolortheme{whale}

\title{Scanning Electrochemical Microscopy (SECM)}
\author{Your Name}
\institute{Your Institution}
\date{\today}

\begin{document}

\begin{frame}
\titlepage
\end{frame}

\begin{frame}
\frametitle{Introduction}
Hello everyone! Today I'm going to introduce you to Scanning Electrochemical Microscopy, or SECM - a powerful technique that allows us to create images of catalytic activity at surfaces.

\vspace{0.5cm}
This technique combines principles from:
\begin{itemize}
    \item Electrochemistry
    \item Catalysis
    \item Imaging
\end{itemize}
\end{frame}

\begin{frame}
\frametitle{Understanding Imaging}
\begin{itemize}
    \item An image is essentially a matrix of numbers
    \item Each number represents a measurement at a specific point
    \item In our case, these numbers will represent electrochemical signals
    \item Just like how a SEM builds an image point by point using electrons, SECM builds an image using electrochemical signals
\end{itemize}
\end{frame}

\begin{frame}
\frametitle{Catalysis Principles}
As demonstrated in our animation:
\begin{itemize}
    \item Without a catalyst, reactions proceed slowly
    \item With a catalyst, the same reaction happens much faster
    \item The catalyst itself remains unchanged
\end{itemize}

\vspace{0.5cm}
\textbf{The Challenge:}\\
When we have a catalytic surface, the products diffuse away in all directions. How can we measure where and how active the catalyst is?
\end{frame}

\begin{frame}
\frametitle{Electrochemistry Fundamentals}
Electrochemical detection works through simple principles:
\begin{itemize}
    \item When chemicals react at an electrode surface, they produce electrical signals
    \item These signals are proportional to the concentration of the reacting species
    \item The higher the concentration, the stronger the signal
\end{itemize}
\end{frame}

\begin{frame}
\frametitle{SECM Working Principle}
\begin{enumerate}
    \item We have a detection electrode scanning above a surface
    \item When it passes over active catalytic areas:
    \begin{itemize}
        \item It detects the products being formed
        \item The signal increases
        \item This creates a peak in our measurement
    \end{itemize}
    \item By scanning systematically:
    \begin{itemize}
        \item We collect signals point by point
        \item These build up into a 2D image
        \item The image reveals where catalytic activity is highest
    \end{itemize}
\end{enumerate}
\end{frame}

\begin{frame}
\frametitle{SECM Visualization}
Our visualization shows four key panels:
\begin{itemize}
    \item The 3D perspective of the scanning process
    \item The side view of electrode movement
    \item The real-time signal response
    \item The resulting 2D activity map
\end{itemize}
\end{frame}

\begin{frame}
\frametitle{Conclusion}
SECM provides a unique way to 'see' chemical reactivity:
\begin{itemize}
    \item Combines electrochemistry, catalysis, and imaging principles
    \item Maps out catalyst activity distribution
    \item Helps design better materials for:
    \begin{itemize}
        \item Fuel cells
        \item Chemical synthesis
        \item And more...
    \end{itemize}
\end{itemize}
\end{frame}

\end{document}